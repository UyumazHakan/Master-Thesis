\chapter{\abstractname}

%TODO: Abstract

In the current technology state of the world, internet acts as the central hub for all kinds of communication between different nodes. Regardless of the subject, all types of software use web services that are built with their own well-defined, high-level and abstract architectural design to accomplish this communication with each other. The architecture that will be used to construct the system communication can differ in terms of needs. Plenty of different architectures and protocols exists to build web services. Nevertheless, the choices of protocols would always be the same for the similar work field since constraints that should be satisfied by the web services will be alike. Moreover, the architecture design of these technologies will also be very similar.

The web services in the Internet of Things (IoT) are also defined in a very similar way and mostly based on REST, CoAP or MQTT. The programmers who are building IoT web services are dealing with implementation of the same functionalities with a little difference under this similarity. The differences mainly occur regarding the device choices and how the data obtained from these devices are processed. However, these differences can be parameterized. Therefore, the programmers will only require to configure their system in a high-level perspective. These configurations will consist of the information about the user devices,  protocols, how data is stored and how data is processed.

The aim of this thesis is to build a rule engine to easily configure required functionalities of web services in the IoT domain. The rule engine will use rules to process and to analyze any data that is received by the web services. A rule can be applied either on data streams to derive the active state or the data batches to process and to analyze historical data of a device or devices. By achieving this goal, an easy-to-use -with high-level configuration- development environment in the domain of IoT will be introduced. These high-level configurations can either be done in a user-friendly web interface or through an application program interface. Therefore, the web services developer can define functionalities regarding the process and the analysis of the data. Also, the challenging gap between high-level design and actual low-level implementation will be covered with the rules that are configured by the users to match their design.

