\chapter{\abstractname}

%TODO: Abstract
In the current technology state of the world, internet acts as the central hub for all kinds of communication between different nodes. Regardless of the subject, all kinds of software use web services that are built with their own well-defined, high-level and abstract web service architectures to accomplish this communication with each other. Architecture that will be used to build the system communication can differ in terms of needs. To transfer the state between two nodes, HTTP-based REST architecture or UDP-based CoAP can be used. If the transfer between two nodes requires any other transport layer protocol or the service discovery is needed for one or more nodes in the architecture, then SOAP would be a better choice. For the systems that require many-to-many communication between nodes, a protocol such as MQTT with the publish-subscribe pattern can be used to build required web services.

Plenty of different architectures and protocols exists to build web services. Nevertheless, the choices would always be the same for the similar work field since constraints that should be satisfied by the web services will be alike. Moreover, the architecture design of these technologies will also be very similar.

The web services in the Internet of Things (IoT) are also defined in a very similar way and mostly based on REST, CoAP or MQTT. The programmers who are building IoT web services are dealing with implementation of the same functionalities with a little difference by virtue of this similarity. However, these differences can be parameterized. Therefore, the programmers will only require to configure their systen in a high-level perpective. These configurations will consist of the information about the used devices, the protocols, how data is stored and how data is proccessed.

The aim of this thesis is to build a Platform as a Service (PaaS) to develop web services that are widely used in the area of IoT in a practical manner without losing any functionality that is needed. By achieving this goal, an easy-to-use - with high-level configuration - development environment in domain of IoT will be introced. Therefore, the developers can establish and maintain the IoT web services in less time without repeating their old works. 

To achieve a handy, agile and generic PaaS; different components that each has different capabilities shall be defined. The primary capabilities of these components can be listed as; the management and storage of the data as ontologies, applying rules on ontologies as needed and the interoperability of these different components. With the help of this PaaS, the developers will able to provide proof of concepts, prototypes or even actual working products by investing a little amount of time in building their web services.
