\chapter{\abstractname}

%TODO: Abstract


In the current technology state of the world, internet acts as a main hub for all kinds of communication between different nodes. Regardless of the subject, the all kind of softwares are using web services build with their respective well-defined, high-level and abstract web service architectures to accomplish this communication between each other. Architecture that will be used to build the system communication can differ in terms of needs. To state transfer between two different node, Representational State Transfer (REST) architecture which is based on HTTP or Constrained Application Protocol (CoAP) which is based on UDP can be used. If the tranfer between two nodes requires any other transport layer protocol other than TCP or UDP and/or the service discovery is required for the one of the nodes, then Simple Object Access Protocol (SOAP) would be a better choice. For the systems that consist of nodes that are depended on data which is sent by more than one node, a protocol such as Message Queuing Telemetry Transport (MQTT) which is based on publish-subscribe pattern can be used to build required web services.

Plenty of different architecures and protocols that can be used to build web services exists. Nevertheless, the choices would always be same for same the work field since constraints that should be satisfied by the web services will be same. Moreover, the use of these technologies will also be very similar.

The web services in Internet of Things (IoT) are also defiend very similar way and mostly based on REST, CoAP and MQTT. The programmers who work to build web services for IoT are dealing with implementing same functionalities with a diffence degree by virtue of this similarity. However, these differences can be parameterized and more practical way for the implementation of the web services can be achieved.

The aim of this thesis is to build a Platform as a Service (PaaS) to develop web services that are used in area of IoT in a practical manner without loosing any functionality that is needed.