
In the current technology state of the world, internet acts as the central hub for all kinds of communication between different nodes. Regardless of the subject, all kinds of software use web services that are built with their own well-defined, high-level and abstract web service architectures to accomplish this communication with each other. Architecture that will be used to build the system communication can differ in terms of needs. To state transfer between two different nodes, Representational State Transfer (REST) architecture which is based on HTTP or Constrained Application Protocol (CoAP) which is based on UDP can be used. If the transfer between two nodes requires any other transport layer protocol or the service discovery is needed for one or more nodes in the architecture, then Simple Object Access Protocol (SOAP) would be a better choice. For the systems that consist of nodes that are depended on data which is sent by more than one node, a protocol such as Message Queuing Telemetry Transport (MQTT) which is based on the publish-subscribe pattern can be used to build required web services.

Plenty of different architectures and protocols exists to build web services. Nevertheless, the choices would always be the same for the similar work field since constraints that should be satisfied by the web services will be alike. Moreover, the stand of these technologies in the architecture will also be very similar.

The web services in the Internet of Things (IoT) are also defined in a very similar way and mostly based on REST, CoAP or MQTT. The programmers who are building web services for IoT are dealing with implementation of the same functionalities with a little difference by virtue of this similarity. However, these differences can be parameterized, and a more practical way for the implementation of the web services can be achieved.

The aim of this thesis is to build a Platform as a Service (PaaS) to develop web services that are widely used in the area of IoT in a practical manner without losing any functionality that is needed. To achieve a handy, agile and generic PaaS; different components which are built on Node.js to take advantage of its light-weight, event-driven and non-blocking model will be proposed such as components that manage data as an ontology model and keep it in Elasticsearch, a rule engine to process the ontology models and the management of interoperability of all these components. With the help of this PaaS, the developers will able to provide proof of concepts, prototypes or even actual working products by investing a little amount of time in building their web services.