\chapter{\abstractname}

%TODO: Abstract


In the current technology state of the world, internet acts as the main hub for all kinds of communication between different nodes. Regardless of the subject, the all kind of software are using web services build with their respective well-defined, high-level and abstract web service architectures to accomplish this communication between each other. Architecture that will be used to build the system communication can differ in terms of needs. To state transfer between two different nodes, Representational State Transfer (REST) architecture which is based on HTTP or Constrained Application Protocol (CoAP) which is based on UDP can be used. If the transfer between two nodes requires any other transport layer protocol other than TCP or UDP and/or the service discovery is needed for one or more nodes in the architecture, then Simple Object Access Protocol (SOAP) would be a better choice. For the systems that consist of nodes that are depended on data which is sent by more than one node, a protocol such as Message Queuing Telemetry Transport (MQTT) which is based on the publish-subscribe pattern can be used to build required web services.

Plenty of different architectures and protocols exists  to be used to build web services. Nevertheless, the choices would always be the same for the similar work field since constraints that should be satisfied by the web services will be alike. Moreover, the stand of these technologies in the architecture will also be very similar.

The web services in the Internet of Things (IoT) are also defined in a very similar way and mostly based on REST, CoAP and MQTT. The programmers who work to build web services for IoT are dealing with implementing the same functionalities with a little difference by virtue of this similarity. However, these differences can be parameterized and a more practical way for the implementation of the web services can be achieved.

The aim of this thesis is to build a Platform as a Service (PaaS) to develop web services that are used in the area of IoT in a practical manner without losing any functionality that is needed. To achieve handy, agile and generic PaaS; different components which are built on Node.js to take advantage of its light-weight, event-driven and non-blocking model will be proposed such as an ontology model to keep data in Elasticsearch, a rule engine to process the ontology models and the management of interoperability of all these components. With help of this PaaS, the developers will able to provide proof of concepts, prototypes or even actual working products by investing little amount of time in building their required web services.